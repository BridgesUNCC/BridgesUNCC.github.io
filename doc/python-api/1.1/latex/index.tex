This is the main superclass in B\+R\+I\+D\+G\+ES for deriving a number of objects used in building arrays, lists, trees and graph data structures. \mbox{\hyperlink{namespace_bridges_1_1_s_lelement}{S\+Lelement}}, \mbox{\hyperlink{namespace_bridges_1_1_d_lelement}{D\+Lelement}}, \mbox{\hyperlink{namespace_bridges_1_1_circ_s_lelement}{Circ\+S\+Lelement}}, \mbox{\hyperlink{namespace_bridges_1_1_circ_d_lelement}{Circ\+D\+Lelement}}, \mbox{\hyperlink{namespace_bridges_1_1_tree_element}{Tree\+Element}}, \mbox{\hyperlink{namespace_bridges_1_1_bin_tree_element}{Bin\+Tree\+Element}}, \mbox{\hyperlink{namespace_bridges_1_1_b_s_t_element}{B\+S\+T\+Element}}, \mbox{\hyperlink{namespace_bridges_1_1_circ_s_lelement}{Circ\+S\+Lelement}}, \mbox{\hyperlink{namespace_bridges_1_1_circ_d_lelement}{Circ\+D\+Lelement}}, \mbox{\hyperlink{namespace_bridges_1_1_a_v_l_tree_element}{A\+V\+L\+Tree\+Element}} are all subclasses (see class hierarchy above). \mbox{\hyperlink{class_bridges_1_1_element_1_1_element}{Element}} contains two visualizer objects (\mbox{\hyperlink{namespace_bridges_1_1_element_visualizer}{Element\+Visualizer}}, \mbox{\hyperlink{namespace_bridges_1_1_link_visualizer}{Link\+Visualizer}}) for specifying visual attributes for nodes and links respectively. It also contains a label that that can be displayed in B\+R\+I\+D\+G\+ES visualizations.

All the tutorials under

\href{http://bridgesuncc.github.io/Hello_World_Tutorials/Overview.html}{\tt http\+://bridgesuncc.\+github.\+io/\+Hello\+\_\+\+World\+\_\+\+Tutorials/\+Overview.\+html}

illustrate examples of using different types of \mbox{\hyperlink{class_bridges_1_1_element_1_1_element}{Element}} objects and how to manipulate their visual attributes.

\begin{DoxyAuthor}{Author}
Mihai Mehedint, Kalpathi Subramanian! 
\end{DoxyAuthor}
\begin{DoxyDate}{Date}
7/18/16 (version 1.\+0), 12/23/18 (version 1.\+1)
\end{DoxyDate}
\hypertarget{index_overview_sec}{}\section{Overview}\label{index_overview_sec}
The Bridging Real-\/world Infrastructure Designed to Goal-\/align, Engage, and Stimulate (B\+R\+I\+D\+G\+ES) project is directed at improving the retention of sophomore students in Computer Science by (1) introducing real-\/world datasets (Facebook, Twitter, I\+M\+DB (Actor/\+Movie), Earthquake, Stock Charts, etc) into course projects involving computer science data structures, and (2) facilitating peer mentoring of sophomores by senior CS students in shared lab experiences that involve the use of the B\+R\+I\+D\+G\+ES infrastructure in software development. \hypertarget{index_br_client}{}\section{B\+R\+I\+D\+G\+E\+S Client Design}\label{index_br_client}
B\+R\+I\+D\+G\+ES client side design loosely follows the basic data structure elements implemented and described in \`{}\`{}A Practical Introduction to Data Structures and Algorithm Analysis" by C.\+A. Shaffer (\href{http://people.cs.vt.edu/shaffer/Book/}{\tt http\+://people.\+cs.\+vt.\+edu/shaffer/\+Book/}). These elements are augmented to contain visual properties that are controlled by the user to customize the visual representation of the constructed data structure. Once a a data structure is ready to be visualized, related B\+R\+I\+D\+G\+ES server calls are made to send a reprsentation of the data structure to B\+R\+I\+D\+G\+ES server. \hypertarget{index_br_server}{}\section{B\+R\+I\+D\+G\+E\+S Server Design.}\label{index_br_server}
B\+R\+I\+D\+G\+ES server implements a combination of technologies (Mongo\+DB, Node, d3\+J\+S(visualization) to receive a data structure representation for visualization. These are largely transparent to the user and involves the user being directed to a web page for viewing the data structure. Attention has been paid to provide meaningful error messages to the user in case problems are encountered in the process. \hypertarget{index_api_sec}{}\section{A\+P\+I Descriptions.}\label{index_api_sec}
See the accompanying pages for detailed description of the B\+R\+I\+D\+G\+ES classes \hypertarget{index_sponsor_sec}{}\section{Sponsorship/\+Funding.}\label{index_sponsor_sec}
B\+R\+I\+D\+G\+ES is funded by the National Science Foundation (an N\+SF T\+U\+ES Project).\hypertarget{index_contacts_sec}{}\section{Contacts\+:}\label{index_contacts_sec}
\char`\"{}\+Kalpathi Subramanian, krs@uncc.\+edu, Jamie Payton, payton@temple.\+edu,
  Erik Saule, esaule@uncc.\+edu, Paula Goolkasian, pagoolka@uncc.\+edu\char`\"{}

Department of Computer Science, The University of North Carolina at Charlotte, Charlotte, NC. 
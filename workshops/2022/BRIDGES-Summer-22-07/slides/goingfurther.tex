\documentclass[aspectratio=169]{beamer}
\usetheme{Madrid}

\title{Going Further}
\subtitle{}
\author{Kalpathi Subramanian, Erik Saule\\\texttt{krs@uncc.edu}, \texttt{esaule@uncc.edu}}
\institute{The University of North Carolina at Charlotte}
\date{BRIDGES Summer Workshop 2022}

\setbeamertemplate{footline}

\usepackage{hyperref,graphicx}
\usepackage{subfloat}

\beamertemplatenavigationsymbolsempty % remove navigation symbols
\begin{document}


\begin{frame}
\titlepage
\end{frame}


\AtBeginSection[]
{
    \begin{frame}
        \frametitle{Table of Contents}
        \tableofcontents[currentsection]
    \end{frame}
}


\begin{frame}
  \frametitle{Build some BRIDGES}
  \begin{block}{Other courses could use BRIDGES}
    Other courses that you teach?

    Other courses one of your colleagues teach?

    If there are multiple people at your institution interested, we can come too!
  \end{block}

  \begin{block}{Attend our monthly meeting}
    Talk to other BRIDGES users about their experience
  \end{block}

  \begin{block}{TA support}
    We can provide some TA support for qualifying courses
  \end{block}
\end{frame}

\begin{frame}
  \frametitle{Help us}
    \begin{block}{Integrate new useful datasets}
      It's easy to find datasets

      It's hard to find datasets that lead to multiple good assignments
    \end{block}

    \begin{block}{Design new assignments}
      Rather than thinking of a cool assignment

      Think of a topic in a class that you want to strenghen

      What's a good assignment for that, hopefully that uses data, visualization      
    \end{block}

    \begin{block}{Documentation}
      Documentation in any form is always useful.
      \begin{itemize}
      \item Underpresented feature
      \item Video tutorials
      \item Better doxygen
      \end{itemize}
    \end{block}
\end{frame}



\begin{frame}
  \frametitle{Plan for next semester}

  \begin{block}{What Remains to be Done?}
    Do you have solutions for all activities?

    Do you have a scaffold for all activities?

    Do you have materials to introduce BRIDGES to students?

    Do you have assignment descriptions?

    Do you have grading criteria?
    
    Do you want more BRIDGES activities?
  \end{block}

  \begin{block}{Make a Precise Plan}
    How long are these tasks?
    
    Who will do it?

    Do you need to train a TA?

    When will you do it?

    Ideally, you'll be done before the semester starts!
  \end{block}
\end{frame}


\begin{frame}
  \frametitle{Use CS Materials to classify your course}

  Thinking of your class in term of modules composed of materials really help understanding the structure of the course.

  You can classify your course at the material level.

  Help answer questions like:
  \begin{itemize}
  \item What am I missing?
  \item Am I over covering some topics?
  \item Do lectures and activities line up well?
  \item What else could I be doing?
  \item Could I use someone else materials?
  \item Can my materials be useful to someone else?
  \end{itemize}
\end{frame}

\end{document}
%
